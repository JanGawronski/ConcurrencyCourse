\documentclass[12pt, a4paper]{article}
\usepackage{fullpage}
\usepackage[T1]{fontenc}
\usepackage{multicol}
\usepackage{amsmath}
\numberwithin{figure}{subsection}
\numberwithin{table}{subsection}
\usepackage{array}
\usepackage{listings}
\lstset{language=Java,
    basicstyle=\ttfamily,
    keywordstyle=\color{blue},
    stringstyle=\color{red},
    commentstyle=\color{green},
    morecomment=[l][\color{magenta}]{\#}
}
\usepackage{enumitem}
\usepackage{graphicx}
\usepackage{float}
\usepackage{adjustbox}
\usepackage{xcolor}
\usepackage{caption}
\renewcommand{\thefigure}{\textbf{\arabic{section}.\arabic{subsection}.\arabic{figure}}}
\renewcommand{\figurename}{\textbf{Rysunek}}
\renewcommand{\thetable}{\textbf{\arabic{section}.\arabic{subsection}.\arabic{table}}}
\renewcommand{\tablename}{\textbf{Tabela}}
\usepackage{hyperref}
\hypersetup{
    colorlinks=true,
    linkcolor=black,
    filecolor=black,      
    urlcolor=black,
    pdftitle={Pomiary porównawcze},
    pdfauthor={Jan Gawroński},
    pdfpagemode=FullScreen,
    }

\title{\huge{Pomiary porównawcze}}
\author{\large{Jan Gawroński}}
\date{18.11.2025}

\begin{document}
  \maketitle

    \section{2 warunki}
    \subsection{Kod}
    \begin{lstlisting}
public class RandomProduceConsume {
    private int buffer = 0;
    private final int bufferMax;
    private final ReentrantLock lock = new ReentrantLock(true);
    private final Condition notFull = lock.newCondition();
    private final Condition notEmpty = lock.newCondition();

    public RandomProduceConsume(int bufferMax) {
        this.bufferMax = bufferMax;
    }   

    public void produce(int amount) {
        lock.lock();
        try {
            while (buffer + amount >= bufferMax)
                notFull.await();
            buffer += amount;
            notEmpty.signal();
        } catch (InterruptedException e) {
            e.printStackTrace();
        } finally {
            lock.unlock();    
        }    
    }

    public void consume(int amount) {
        lock.lock();
        try {
            while (buffer - amount < 0)
                notEmpty.await();
            buffer -= amount;
            notFull.signal();
        } catch (InterruptedException e) {
            e.printStackTrace();
        } finally {
            lock.unlock();        
        }
    }
}
    \end{lstlisting}

    \subsection{Wykresy}
    \begin{figure}[H]
        \includegraphics[width=\textwidth]{2_conditionbuffer40.png}
    \end{figure}

    \begin{figure}[H]
        \includegraphics[width=\textwidth]{2_conditionbuffer400.png}
    \end{figure}

    \section{4 warunki}
    \subsection{Kod}
    \begin{lstlisting}
ublic class CorrectRandomProduceConsume {
    private int buffer = 0;
    private final int bufferMax;
    private final ReentrantLock lock = new ReentrantLock(true);
    private final Condition firstProd = lock.newCondition();
    private final Condition restProd = lock.newCondition();
    private final Condition firstCons = lock.newCondition();
    private final Condition restCons = lock.newCondition();

    private boolean isFirstProducerWaiting = false;
    private boolean isFirstConsumerWaiting = false;

    public CorrectRandomProduceConsume(int bufferMax) {
        this.bufferMax = bufferMax;
    }   

    public void produce(int amount) {
        lock.lock();
        try {
            while (isFirstProducerWaiting)
                restProd.await();
            isFirstProducerWaiting = true;
            while (buffer + amount >= bufferMax)
                firstProd.await();
            buffer += amount;
            isFirstProducerWaiting = false;
            restProd.signal();
            firstCons.signal();
        } catch (InterruptedException e) {
            e.printStackTrace();
        } finally {
            lock.unlock();    
        }    
    }

    public void consume(int amount) {
        lock.lock();
        try {
            while (isFirstConsumerWaiting)
                restCons.await();
            isFirstConsumerWaiting = true;
            while (buffer - amount < 0)
                firstCons.await();
            buffer -= amount;
            isFirstConsumerWaiting = false;
            restCons.signal();
            firstProd.signal();
        } catch (InterruptedException e) {
            e.printStackTrace();
        } finally {
            lock.unlock();    
        }
    }
}
    \end{lstlisting}

    \subsection{Wykresy}
    \begin{figure}[H]
        \includegraphics[width=\textwidth]{4_conditionbuffer40.png}
    \end{figure}

    \begin{figure}[H]
        \includegraphics[width=\textwidth]{4_conditionbuffer400.png}
    \end{figure}

    \section{Zagnieżdżone blokady}
    \subsection{Kod}
    \begin{lstlisting}
public class NestedLocks {
    private int buffer = 0;
    private final int bufferMax;
    private final ReentrantLock prodLock = new ReentrantLock(true);
    private final ReentrantLock consLock = new ReentrantLock(true);
    private final ReentrantLock lock = new ReentrantLock();
    private final Condition prod = lock.newCondition();
    private final Condition cons = lock.newCondition();


    public NestedLocks(int bufferMax) {
        this.bufferMax = bufferMax;
    }   

    public void produce(int amount) {
        prodLock.lock();
        lock.lock();
        try {
            while (buffer + amount >= bufferMax)
                prod.await();
            buffer += amount;
            cons.signal();
        } catch (InterruptedException e) {
            e.printStackTrace();
        } finally {
            lock.unlock();
            prodLock.unlock();    
        }    
    }

    public void consume(int amount) {
        consLock.lock();
        lock.lock();
        try {
            while (buffer - amount < 0)
                cons.await();
            buffer -= amount;
            prod.signal();
        } catch (InterruptedException e) {
            e.printStackTrace();
        } finally {
            lock.unlock();
            consLock.unlock();    
        }    
    }
}

    \end{lstlisting}

    \subsection{Wykresy}
    \begin{figure}[H]
        \includegraphics[width=\textwidth]{nestedlocksbuffer40.png}
    \end{figure}

    \begin{figure}[H]
        \includegraphics[width=\textwidth]{nestedlocksbuffer400.png}
    \end{figure}

    \section{Wnioski}
    Zwiększony bufor zwiększa przepustowość zgodnie z oczekiwaniami.\\
    Najszybszym rozwiązaniem okazały się zagnieżdżone blokady, wynika to z tego, że blokady to szybszy mechanizm niż warunki.\\

\end{document}