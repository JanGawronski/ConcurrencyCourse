\documentclass[12pt, a4paper]{article}
\usepackage{fullpage}
\usepackage[T1]{fontenc}
\usepackage{multicol}
\usepackage{amsmath}
\numberwithin{figure}{subsection}
\numberwithin{table}{subsection}
\usepackage{array}
\usepackage{enumitem}
\usepackage{graphicx}
\usepackage{float}
\usepackage{placeins}
\usepackage{adjustbox}
\usepackage{xcolor}
\usepackage{caption}
\usepackage{pgfplots}
\pgfplotsset{compat=1.18}
\usepgfplotslibrary{groupplots}
\usepackage{pgfplotstable}
\usepackage{hyperref}
\hypersetup{
    colorlinks=true,
    linkcolor=black,
    filecolor=black,      
    urlcolor=black,
    pdftitle={Sieci Petriego},
    pdfauthor={Jan Gawroński},
    pdfpagemode=FullScreen,
    }

\title{\huge{Sieci Petriego}}
\author{\large{Jan Gawroński}}
\date{20.01.2026}

\begin{document}
    \maketitle
    \section{Zadanie 1}
      \begin{figure}[H]
          \centering
          \includegraphics[width=\textwidth]{1/net.png}
      \end{figure}
      \begin{figure}[H]
          \centering
          \includegraphics[width=0.8\textwidth]{1/sim.png}
      \end{figure}
      \begin{figure}[H]
          \centering
          \includegraphics[width=0.5\textwidth]{1/reach.png}
      \end{figure}
      \begin{figure}[H]
          \centering
          \includegraphics[width=0.8\textwidth]{1/invariant.png}
      \end{figure}
    \section{Zadanie 2}
      \begin{figure}[H]
          \centering
          \includegraphics[width=0.8\textwidth]{2/net.png}
      \end{figure}
      \begin{figure}[H]
          \centering
          \includegraphics[width=0.5\textwidth]{2/invariant.png}
      \end{figure}
      Sieć nie jest pokryta niezmiennikami, więc nie jest odwracalna.
      \begin{figure}[H]
          \centering
          \includegraphics[width=0.5\textwidth]{2/reach.png}
      \end{figure}
      Sieć jest żywa. Sieć cały czas przechodzi przez cykl $T_0, T_1, T_2$, używając po koleji każdego z przejść.

      Sieć nie jest ograniczona, ponieważ w miejscu $P_3$ może się nagromadzić dowolna liczba znaczników.

    \section{Zadanie 3}
      \begin{figure}[H]
          \centering
          \includegraphics[width=0.8\textwidth]{3/net.png}
      \end{figure}
      \begin{figure}[H]
          \centering
          \includegraphics[width=0.5\textwidth]{3/invariant.png}
      \end{figure}
      Równania niezmnieników miejc pokazują ile znaczników zawsze bedzię w jakiejś grupie miejsc.
      Równanie $M(P_0) + M(P_1) + M(P_2) = 1$ to równanie sekcji krytycznej.
    
    \section{Zadanie 4}
      \begin{figure}[H]
          \centering
          \includegraphics[width=0.8\textwidth]{4/net.png}
      \end{figure}
      \begin{figure}[H]
          \centering
          \includegraphics[width=0.5\textwidth]{4/invariant.png}
      \end{figure}
      Sieć jest zachowawcza, widać to po 3 równaniach niezmienników miejsc, każde z tych równań ma rozłączne składniki.
      Równanie $M(P_6) + M(P_7) = 3$ mówi nam o rozmiarze bufora.
    
    \section{Zadanie 5}
      \begin{figure}[H]
          \centering
          \includegraphics[width=0.8\textwidth]{5/net.png}
      \end{figure}
      \begin{figure}[H]
          \centering
          \includegraphics[width=0.5\textwidth]{5/invariant.png}
      \end{figure}
    
    \section{Zadanie 6}
      \begin{figure}[H]
          \centering
          \includegraphics[width=0.8\textwidth]{6/net.png}
      \end{figure}
      \begin{figure}[H]
          \centering
          \includegraphics[width=0.5\textwidth]{6/reach.png}
      \end{figure}
      \begin{figure}[H]
          \centering
          \includegraphics[width=0.5\textwidth]{6/state.png}
      \end{figure}
    \section{Zadanie 7}
      \begin{figure}[H]
          \centering
          \includegraphics[width=\textwidth]{1/net.png}
      \end{figure}
      \begin{figure}[H]
          \centering
          \includegraphics[width=0.8\textwidth]{1/reach.png}
      \end{figure}
      \begin{figure}[H]
          \centering
          \includegraphics[width=0.6\textwidth]{1/state.png}
      \end{figure}
    
    \section{Zadanie 8}
      \begin{figure}[H]
          \centering
          \includegraphics[width=0.8\textwidth]{8/net.png}
      \end{figure}
      \begin{figure}[H]
          \centering
          \includegraphics[width=0.5\textwidth]{8/reach.png}
      \end{figure}
      \begin{figure}[H]
          \centering
          \includegraphics[width=0.8\textwidth]{8/invariant.png}
      \end{figure}

      \end{document}
