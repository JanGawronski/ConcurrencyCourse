\documentclass[12pt, a4paper]{article}
\usepackage{fullpage}
\usepackage[T1]{fontenc}
\usepackage{multicol}
\usepackage{amsmath}
\numberwithin{figure}{subsection}
\numberwithin{table}{subsection}
\usepackage{listings}
\lstset{
  language=Java,
  basicstyle=\ttfamily\small,
  keywordstyle=\color{blue}\bfseries,
  commentstyle=\color{gray},
  stringstyle=\color{green!50!black},
  breaklines=true,
  numbers=left,
  numberstyle=\tiny,
  frame=single,
  captionpos=b
}
\usepackage{array}
\usepackage{enumitem}
\usepackage{graphicx}
\usepackage{float}
\usepackage{adjustbox}
\usepackage{xcolor}
\usepackage{caption}
\renewcommand{\thefigure}{\textbf{\arabic{section}.\arabic{figure}}}
\renewcommand{\figurename}{\textbf{Rysunek}}
\renewcommand{\thetable}{\textbf{\arabic{section}.\arabic{table}}}
\renewcommand{\tablename}{\textbf{Tabela}}
\usepackage{hyperref}
\hypersetup{
    colorlinks=true,
    linkcolor=black,
    filecolor=black,      
    urlcolor=black,
    pdftitle={Producenci i konsumenci bez zagłodzenia},
    pdfauthor={Jan Gawroński},
    pdfpagemode=FullScreen,
    }

\title{\huge{Teoria współbieżności} \break \Large{Producenci i konsumenci bez zagłodzenia}}
\author{\large{Jan Gawroński}}
\date{12.11.2025}

\begin{document}
    \maketitle
    \section{Poprawny kod (4 warunki + zmnienna logiczna)}
        \lstinputlisting{CorrectRandomProduceConsume.java}
    \section{Zagłodzenie na 2 warunkach}
        \begin{figure}[H]
            \centering
            \includegraphics[width=\textwidth]{2cond.png}
        \end{figure}

        Jak do tego dochodzi:
        
        bufferMax = 10
        \begin{enumerate}
            \item Bufer jest pełny.
            \item Wchodzi P1, który chce wyprodukować 4 i wiesza się na notFull.
            \item Wchodzi P2, który chce wyprodukować 1 i wiesza się na notFull.
            \item Wchodzi K1, konsumuje 2 i budzi P1.
            \item P1 budzi się i wiesza się na notFull.
            \item Wchodzi K1, konsumuje 2 i budzi P1.
            \item P1 budzi się i produkuje 4.
            \item Wchodzi P1, który chce wyprodukować 4 i wiesza się na notFull.
        \end{enumerate}
        Taka sytuacja może się powtarzać w kółko, prowadząc do zagłodzenia P2.

    \section{Zakleszczenie i zagłodzenie na 4 warunkach z hasWaiters()}
        bufferMax = 10
        \begin{enumerate}
            \item K1, który chce skonsumować 3 zawiesza się na firstCons
            \item P1 produkuje 1 i wywołuje firstCons budząc K1
            \item K2, który chce skonsumować 3 wiesza się na firstCons (Tutaj dochodzi do zagłodzenia, ponieważ K1 może już nigdy nie zostać wybudzony)
            \item P1 produkuje 1 i wywołuje firstCons budząc K2
            \item K3, który chce skonsumować 3 wiesza się na firstCons
            \item K1 wiesza się na firstCons
            \item K2 wiesza się na firstCons
            \item K4, który chce skonsumować 1 wiesza się na restCons
            \item P2 produkuje 5 i wywołuje firstCons budząc K2
            \item K2 konsumuje 3 i wywołuje restCons budząc K4
            \item K4 wiesza się na restCons
            \item P2 produkuje 5 i wywołuje firstCons budząc K3
            \item K3 konsumuje 3 i wywołuje restCons budząc K4
            \item K4 wiesza się na restCons
            \item P1 wiesza się na firstProd
            \item P2 - Pp wieszają się na restProd
            \item K2 wiesza się na restCons
            \item K3 wiesza się na restCons
        \end{enumerate}


\end{document}
